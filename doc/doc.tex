\documentclass{article}

% if you need to pass options to natbib, use, e.g.:
%     \PassOptionsToPackage{numbers, compress}{natbib}
% before loading neurips_2021

% ready for submission
\usepackage[preprint]{neurips_2021}

% to compile a preprint version, e.g., for submission to arXiv, add add the
% [preprint] option:
%     \usepackage[preprint]{neurips_2021}

% to compile a camera-ready version, add the [final] option, e.g.:
%     \usepackage[final]{neurips_2021}

% to avoid loading the natbib package, add option nonatbib:
%    \usepackage[nonatbib]{neurips_2021}

\usepackage[utf8]{inputenc} % allow utf-8 input
\usepackage[T1]{fontenc}    % use 8-bit T1 fonts
\usepackage[colorlinks=true]{hyperref}       % hyperlinks
\usepackage{url}            % simple URL typesetting
\usepackage{booktabs}       % professional-quality tables
\usepackage{amsfonts}       % blackboard math symbols
\usepackage{nicefrac}       % compact symbols for 1/2, etc.
\usepackage{microtype}      % microtypography
\usepackage{xcolor}         % colors

\title{Toward an understanding of salary distributions in remote jobs}

% The \author macro works with any number of authors. There are two commands
% used to separate the names and addresses of multiple authors: \And and \AND.
%
% Using \And between authors leaves it to LaTeX to determine where to break the
% lines. Using \AND forces a line break at that point. So, if LaTeX puts 3 of 4
% authors names on the first line, and the last on the second line, try using
% \AND instead of \And before the third author name.

\author{%
  Tim Schreier\\
  Matrikelnummer: 5634978\\
  \texttt{tim.schreier@student.uni-tuebingen.de} \\
  \And
  Luca Schaal\\
  Matrikelnummer: 6052577\\
  \texttt{luca.schaal@student.uni-tuebingen.de} \\
}

\begin{document}

\maketitle

\begin{abstract}We are going to analyse the \href{https://salaries.freshremote.work/download/}{dataset of global remote work salaries}.
We will create visualizations in order to draw insights from the data about how factors like job title or company size contritubute to a persons salary.
Furthermore we will state our hypotheses about how certain factors contribute and test these hypotheses using statistics.
In order to understand how factors are specific facots are correlated with salary, we will use linear regression.
\end{abstract}

\section{The dataset}
The dataset of global remote work saleries which is used in this paper contains 1507 entries. The data set includes salary data from professionals around the world in the field of remote work. There are various features, such as experience level, employment type, company size, company location or remote ratio. 
\end{document}
